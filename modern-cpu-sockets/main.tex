% Copyright (c) 2022 Tobias Briones. All rights reserved.
%
% SPDX-License-Identifier: CC-BY-4.0
%
% This file is part of Course Project at UNAH-IS911: Microprocessors.
%
% This source code is licensed under the Creative Commons Attribution 4.0 
% International License found in the LICENSE file in the root directory of 
% this source tree or at https://spdx.org/licenses/CC-BY-4.0.html.

\documentclass[conference]{IEEEtran}
\usepackage[letterpaper, portrait, margin=2cm]{geometry}
\usepackage[style=ieee]{biblatex}
\usepackage[utf8]{inputenc}
\usepackage[spanish]{babel}
\usepackage{geometry}
\usepackage[]{graphics}
\usepackage[demo]{graphicx}
\usepackage{csquotes}
\usepackage{float}
\usepackage{hyperref}
\usepackage[table,xcdraw]{xcolor}

\addbibresource{bibliography.bib}

\title{SOCKETS DE MICROPROCESADORES DE PC MODERNOS}
\author{
\includegraphics[width = 40mm]{images/logo-unah.png}\\[8ex]
\IEEEauthorblockN{Tobias Briones}
\IEEEauthorblockN{tobias.briones@unah.hn}
\IEEEauthorblockA{\textit{Universidad Nacional Autónoma de Honduras} \\
\textit{Ingeniería de Sistemas} \\
\textit{I PAC 2022} \\
\textit{IS911-MICROPROCESADORES}} \\\vspace*{20pt} \normalsize  \\
\today
}

\hypersetup{
    colorlinks=true,
    linkcolor=black,
    filecolor=magenta,      
    urlcolor=cyan,
    citecolor=black
}
    
\begin{document}

\maketitle

\begin{abstract}
En esta investigación se provee un resumen sobre los sockets para PC modernos año 2022.
\end{abstract}

\tableofcontents

\section{Introducción}

Los sockets son conectores que permiten la conectividad física y eléctrica del microprocesador a la tarjeta madre. Se le denominará a los sockets o slots como zócalos ya que esa es su equivalencia en el español.

\bigbreak

Como se mencionó arriba, podemos afirmar sobre los zócalos el siguiente enunciado:

\begin{displayquote}
    Un zócalo de CPU contiene uno o más componentes mecánicos que proporcionan conexiones mecánicas y eléctricas entre un microprocesador y una placa de circuito impreso (PCB). Esto permite colocar y reemplazar la unidad central de procesamiento (CPU) sin soldar.\\
    \small Fuente: Wikipedia $\mid$ CPU socket \cite{wikipedia-contributors-2022}
\end{displayquote}

\bigbreak

Algunos tipos de sockets, por mencionar, tenemos \cite{authortechnews-2020}: TR4, AM4, LGA 1151, 2066, sTRX4. Al momento de diseñar una configuración para un PC se debe tener en cuenta que la tarjeta madre sea compatible con el resto del hardware escogido y en particular, el socket sea el que corresponde al modelo del CPU que se deberá instalar.

\subsection{Definición}

Una definición simple de un socket es:

\begin{displayquote}
    El \textbf{socket/slot de CPU} o \textbf{zócalo de CPU} se refiere a un conector físico en la placa base de una computadora que acepta un solo chip físico. Muchas placas base pueden tener varios sockets que, a su vez, pueden aceptar chips de varios núcleos.\\
    \small Fuente: University Information Technology Services \cite{university-information-technology-services-2019}
\end{displayquote}

Los ordenadores personales (incluyendo PCs gaming) e industriales cuentan con sockets similares o iguales. Los sockets son mayormente encontrados en estos equipos a diferencia de dispositivos móviles como portátiles o teléfonos inteligentes los cuales traen el CPU soldado en la tarjeta madre o cuentan con un SoC (System on a Chip) el cual es un integrado que implementa el CPU, memoria y demás componentes en un solo chip.

\begin{figure}[H]
    \centering
    \includegraphics[width=0.3\paperwidth]{images/cpu-socket.jpg}
    \caption{Zócalo de CPU} \footnotesize
    Fuente: Imágen por \href{https://pixabay.com/users/bru-no-1161770}{Bruno /Germany} de \href{https://pixabay.com}{Pixabay} \cite{pixabay-cpu-socket-2019}.
\end{figure}

\section{Tipos de Zócalos}

De acuerdo a datos actuales, se van a presentar los tipos de zócalos que se utilizan en la actualidad para computadoras personales, gaming e incluso también workstation o industriales/servidores.

\bigbreak

Existe una gran cantidad de zócalos viejos \footnote{Wikipedia \cite{wikipedia-contributors-2022} enlista una gran cantidad de modelos incluyendo los viejos introducidos desde el año $1,970$. Si el lector es suficientemente viejo podrá reconocer algunos modelos como el $LGA 775$ en el cual funcionaba nuestros viejos Intel Pentium 4/D o Core2Duo.}, se detallarán solo los más recientes.

\bigbreak

Para escritorio PC se encuentran los siguientes modelos:

\begin{table}[H]
\centering
\tiny
\begin{tabular}{|l|l|l|l|l|}
\hline
\rowcolor[HTML]{CBCEFB} 
\textbf{\begin{tabular}[c]{@{}l@{}}Nombre de \\ Zócalo\end{tabular}} & \textbf{\begin{tabular}[c]{@{}l@{}}Año de \\ Introducción\end{tabular}} & \textbf{CPUs}                                                                                              & \textbf{Paquete} & \textbf{Pines} \\ \hline
Socket AM5                                                           & 2022                                                                    & AMD Zen 4                                                                                                  & LGA              & 1718           \\ \hline
\rowcolor[HTML]{EFEFEF} 
LGA 1700                                                             & 2021                                                                    & Intel Alder Lake                                                                                           & LGA              & 1700           \\ \hline
LGA 1200                                                             & 2020                                                                    & \begin{tabular}[c]{@{}l@{}}Intel Comet Lake\\ Intel Rocket Lake\end{tabular}                               & LGA              & 1200           \\ \hline
\rowcolor[HTML]{EFEFEF} 
\begin{tabular}[c]{@{}l@{}}Socket sTRX4/\\ Socket SP3r3\end{tabular} & 2019                                                                    & \begin{tabular}[c]{@{}l@{}}AMD Ryzen \\ Threadripper\\ (series 3,000)\end{tabular}                         & LGA              & 4094           \\ \hline
\begin{tabular}[c]{@{}l@{}}LGA 2066/Socket\\ R4\end{tabular}         & 2017                                                                    & \begin{tabular}[c]{@{}l@{}}Intel Skylake-X\\ Intel Kaby Lake-X\\ Intel Cascade Lake-X\end{tabular}         & LGA              & 2066           \\ \hline
\rowcolor[HTML]{EFEFEF} 
\begin{tabular}[c]{@{}l@{}}Socket TR4/\\ Socket SP3r2\end{tabular}   & 2017                                                                    & \begin{tabular}[c]{@{}l@{}}AMD Ryzen \\ Threadripper\end{tabular}                                          & LGA              & 4094           \\ \hline
Socket AM4                                                           & 2017                                                                    & \begin{tabular}[c]{@{}l@{}}AMD Ryzen 9\\ AMD Ryzen 7\\ AMD Ryzen 5\\ AMD Ryzen 3\\ Athlon 200\end{tabular} & PGA              & 1331           \\ \hline
\end{tabular}
\caption{Zócalos de CPU recientes}
\small
Fuente: Wikipedia \cite{wikipedia-contributors-2022}.
\end{table}

Los ordenadores se pueden clasificar en convencionales (PC o computador personal), HEDT (High-End Desktop son PC de alto rendimiento), o Servidores/Industriales/Estación de Trabajo. Los zócalos que se listaron son para PCs convencionales de escritorio con CPUs muy populares y algunos de alto rendimiento como los AMD Ryzen™ Threadripper™. Los Intel® Core™ X (LGA 2066/Socket R4) son un ejemplo de modelos para escritorio y para servidor también.

\bigbreak

Entre los modelos mencionados arriba y otros populares se resume la siguiente información más práctica que muestra el modelo de CPU(s), generación, chipsets compatibles y tipo de ordenador personal:

\bigbreak

\begin{itemize}
    \item \textbf{Intel LGA 2066:} 10ma Gen., X299, HEDT.
    \item \textbf{Intel LGA 1200:} 11/10ma Gen., Z490/H470, B460, H410, Convencional.
    \item \textbf{Intel LGA 1151:} 9/8va Gen., Z390/Z370/Z370/Q370/H370/B365/B360/H310, Convencional.
    \item \textbf{AMD sTRX4:} Ryzen Threadripper 3000, TRX40, HEDT.
    \item \textbf{AMD TR4:} Ryzen Threadripper 2000 y 1000, x399, HEDT.
    \item \textbf{AMD AM4:} Ryzen 5000, 3000, 2000 y 1000, X570/X470/X370/B550/B450/B350/B450/A320/X300/A300, Convencional.
\end{itemize}

\small Fuente: tom'sHARDWARE \cite{harding-2021}.

\section{Funcionamiento}

De acuerdo al paquete del zócalo, se define su mecanismo de funcionamiento. Como se ha mencionado, el funcionamiento de un zócalo es tal que permite la conexión física del microprocesador a la tarjeta madre.

\subsection{Matriz de Contactos en Rejilla (LGA)}

Los zócalos LGA son los más populares ya que como se ha visto anteriormente, muchos modelos de CPUs actuales para PCs normales lo utilizan.

\bigbreak

Según ONLOGIC Blog \cite{fanton-2021} la mayoría de procesadores modernos removibles utilizan zócalo LGA.

\bigbreak

Su funcionamiento consiste en:

\bigbreak

\begin{displayquote}
    A diferencia de las interfaces de matriz de rejilla de pines (PGA) y matriz de rejilla de bolas (BGA), la interfaz LGA no presenta ni pines ni esferas, la conexión de la que dispone el chip es únicamente una matriz de superficies conductoras o contactos chapadas en oro que hacen contacto con la placa base a través del zócalo de CPU.

    \bigbreak

    Su alineación de pines es vertical y horizontal.
    
    \bigbreak
    
    Esta interfaz se beneficia por reducir el proceso de fabricación, amén de unas características térmicas, eléctricas y físicas superiores a las interfaces de chips previamente usados.\\
    \small
    Fuente: Wikipedia $\mid$ Land Grid Array \cite{wikipedia-lga-2021D}.
\end{displayquote}

\begin{figure}[H]
    \centering
    \includegraphics[width=0.2\paperwidth]{images/lga-socket.jpg}
    \caption{Zócalo de Matriz de Contactos en Rejilla} \footnotesize
    Fuente: De User Smial on de.wikipedia - Trabajo propio, CC BY-SA 2.0 de, https://commons.wikimedia.org/w/index.php?curid=1066500 \cite{wikipedia-lga-2021D}.
\end{figure}

\bigbreak

Como se puede notar, ambos fabricantes mayores, AMD e Intel lo emplean para muchos de sus microprocesadores. Notar que, por ejemplo, el Socket sTRX4 es en realidad un LGA 4094 \cite{wikipedia-lga-2021D}.

\subsection{Matriz de Rejilla de Pines (PGA)}

Su funcionamiento consiste en:

\bigbreak

\begin{displayquote}
    En un PGA, el IC se monta en una losa de cerámica, que presenta una matriz de contactos o olas en una de sus caras. Luego, los pines se pueden insertar en los agujeros de un circuito impreso y soldarse.
    \small
    Fuente: Wikipedia $\mid$ Pin grid array \cite{wikipedia-pga-2021C}.
\end{displayquote}

\subsection{Matriz de Rejilla de Bolas (BGA)}

Estos son más que todo usados para móviles, portátiles o casos de uso en ingeniería eléctrica \cite{wikipedia-bga-2021B} por lo que no nos interesan mucho en este artículo. Tenemos que:

\begin{displayquote}
    Son usadas comúnmente en la producción y fijación de placas base para computadoras y la fijación de microprocesadores ya que los mismos suelen tener una cantidad muy grande de terminales los cuales son soldados a conciencia a la placa base para evitar la pérdida de frecuencias y aumentar la conductividad de los mismos.
    \small
    Fuente: Wikipedia $\mid$ Ball grid array \cite{wikipedia-bga-2021B}.
\end{displayquote}

\subsection{Cero Fuerza de Inserción (ZIF)}

El zócalo ZIF (Zero Insertion Force) permite insertar el CI de forma muy fácil. Se ve por ejemplo, en los programadores de PIC donde el microcontrolador se inserta directamente en el socket.

\begin{figure}[H]
    \centering
    \includegraphics[width=0.2\paperwidth]{images/zif-socket.jpg}
    \caption{Zócalo de Cero Fuerza de Insersión} \footnotesize
    Fuente: Por usuario Smial en de.wikipedia - Trabajo propio, CC BY-SA 2.0 de, https://commons.wikimedia.org/w/index.php?curid=1058788 \cite{wikipedia-zif-2021}.
\end{figure}

\bigbreak

Su funcionamiento consiste en:

\bigbreak

\begin{displayquote}
    Fuerza de inserción cero (ZIF por sus siglas en inglés) es un tipo de zócalo para CI o conector eléctrico que requiere muy poca fuerza para la inserción. Con un zócalo ZIF, antes de insertar el CI, se mueve una palanca o deslizador en el costado del zócalo, separando todos los contactos suspendidos para que el CI pueda insertarse con muy poca fuerza; por lo general, el peso del CI en sí es suficiente y no se requiere ninguna fuerza externa hacia abajo. Luego, la palanca se mueve hacia atrás, lo que permite que los contactos se cierren y agarren los pines del CI. Los zócalos ZIF son mucho más caros que los zócalos de CI estándar y también tienden a ocupar un área de placa más grande debido al espacio que ocupa el mecanismo de palanca. Por lo general, solo se usan cuando hay una buena razón para hacerlo.
    \small
    Fuente: Wikipedia $\mid$ Zero Insertion Force \cite{wikipedia-zif-2021}.
\end{displayquote}

\bigbreak

Por lo que también son zócalos de nicho y no nos interesa mucho por aquí.

\section{Conclusión}

Se enlisto la información principal sobre los zócalos empleados en la actualidad año 2022 para microprocesadores convencionales o PCs. Se explicó brevemente como funcionan los zócalos de CPU y que hay categorías de PCs en función de las necesidades de los usuarios, las cuales requieren de distintos tipos de zócalos. Se observó que el paquete de zócalo LGA es el más utilizado para PCs.

\printbibliography

\end{document}
