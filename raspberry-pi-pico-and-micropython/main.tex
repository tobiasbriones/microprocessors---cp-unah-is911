% Copyright (c) 2022 Tobias Briones. All rights reserved.
% SPDX-License-Identifier: CC-BY-SA-4.0
%
% This source code is part of
% https://github.com/tobiasbriones/cp-unah-is911-microprocessors and is
% licensed under the Creative Commons Attribution Share Alike 4.0
% International License found in the LICENSE file in the root
% directory of this source tree or at https://spdx.org/licenses/CC-BY-SA-4.0

\documentclass{article}
\usepackage{preamble}

\title{LAB 4: RASPBERRY PI PICO Y MICROPYTHON}
\author{Tobias Briones \bigbreak tobias.briones@unah.hn}
\date{\today}

\begin{document}

    \makeatletter
    \begin{titlepage}
        \begin{center}
            \includegraphics[width=0.3\linewidth]{images/logo-unah}\\[4ex]
            {\huge \bfseries \@title
            \vspace{1cm}}\\[2ex]
            {\LARGE \@author}\\[50ex]

            {\large
            Universidad Nacional Autónoma de Honduras\\
            Ingeniería de Sistemas\\
            I PAC 2022\\
            IS911-MICROPROCESADORES
            }\\[2ex]

            {\large \today}
        \end{center}
    \end{titlepage}
    \makeatother
    \thispagestyle{empty}
    \newpage

    \import{}{footer}

    \section{Objetivo}

    Desarrollar ejemplos de programas básicos para simular en Raspberry Pi
    Pico y MicroPython.

    \subsection{Objetivos Específicos}\label{subsec:objetivos-específicos}

    \begin{itemize}
        \item Utilizar un simulador en línea que permita hacer los proyectos.
        \item Entender la tarjeta Raspberry Pi Pico tanto en programación
        como en circuito.
        \item Desarrollar $3$ ejercicios prácticos en total.
    \end{itemize}

    \section{Marco Teórico}\label{sec:marco-teórico}


    Para llevar a cabo este laboratorio se necesita conocer sobre Raspberry
    Pi Pico, MicroPython y de alguna herramienta para crear la simulación.

    \subsection{Resistencia Pull Down}

    Las resistencias Pull-Up o Pull-Down sirven para que la entrada sea
    definida correctamente. Tener nada conectado a un pin no significa que la
    señal sea cero, por tanto una resistencia Pull-Down es una resistencia
    Pull-Up conectada a tierra que verifica que la señal sea cero cuando no
    hay conectado otro dispositivo. Su valor depende de lo que de va a
    conectar \cite{arduino-2018}.

    \subsection{Raspberry Pi Pico}

    Esta es una versión simple y económica de Raspberry, es una tarjeta
    microcontrolador muy similar a Arduino. Actualmente su precio es de $4\$$
    . Un review de esta tarjeta y su precio se puede consultar en
    \href{https://www.raspberrypi.com/products/raspberry-pi-pico}{Buy a
    Raspberry Pi Pico - Raspberry Pi}. La documentación se puede consultar en
    \href{https://www.raspberrypi.com/documentation/microcontrollers/raspberry-pi-pico.html}{Raspberry Pi
    Documentation - Raspberry Pi Pico} y se cuentan con los siguientes datos
    \cite{raspberry-pi-ltd-2022}:

    \begin{itemize}
        \item Microcontrolador RP2040.
        \item Procesador doble núcleo ARM Cortex M0+, a 133MHz.
        \item 264KB de SRAM, y 2MB de memoria flash.
        \item USB 1.1.
        \item $26$ $\times$ GPIO pines multifuncionales.
        \item $2$ $\times$ SPI, $2 \times I2C$, $2 \times UART$, $3 \times
        12-bit ADC$, $16 \times $ canales $PWM$ controlables.
        \item Sensor de temperatura.
        \item Otros.
    \end{itemize}

    La disposición de la tarjeta se ve en la siguiente imagen:

    \begin{figure}[H]
        \centering
        \includegraphics[width=0.5\paperwidth]{images/pico-r3-sdk11-pinout}
        \caption{Raspberry Pi Pico}\footnotesize
        Fuente: \textit{Raspberry Pi Documentation} licensed under the CC
        BY-SA 4.0 Licence \cite{raspberry-pi-ltd-2022}
    \end{figure}

    \subsection{MicroPython}

    MicroPython es una versión ligera de Python para programar controladores
    disponible bajo la licencia MIT. Según la descripción oficial:

    \begin{quote}
        MicroPython es una implementación sencilla y eficiente del lenguaje
        de programación Python 3 que incluye un pequeño subconjunto de la
        biblioteca estándar de Python y está optimizado para ejecutarse en
        microcontroladores y en entornos restringidos.\\ \footnotesize
        Fuente: \textit{MicroPython - Python for microcontrollers} (bajo uso
        justo, traducido de inglés a español)
    \end{quote}

    \subsection{Wokwi}

    Wokwi es una aplicación web de código abierto que permite crear proyectos
    de simulación de circuitos de sistemas embebidos. Actualmente es muy
    limitada pero es una excelente herramienta teniendo en cuenta que es
    moderna y se está actualizando. Por la comunidad y lo que nos ofrece es
    definitivamente una plataforma recomendable. Además también hay soporte
    para uso profesional \cite{codemagic-ltd-2022}.

    \bigbreak

    Tenemos oficialmente que:

    \begin{quote}
        Wokwi es un simulador de electrónica en línea. Puede usarlo para
        simular Arduino, ESP32 y muchas otras placas, piezas y sensores
        populares.
        \\ \footnotesize
        Fuente: \textit{Welcome to Wokwi! $\mid$ Wokwi Docs} bajo la licencia
        CC BY 4.0
    \end{quote}

    \section{Procedimiento Experimental}\label{sec:procedimiento-experimental}

    Para hacer los experimentos se usa \href{https://wokwi.com}{Wokwi}. En
    este se desarrollará el programa como el circuito y su simulación para
    cada ejercicio~\cite{ricardo-adonis-caraccioli-abrego-2022}.

    \subsection{Ejemplo 1}

    Desarrollar un programa en MicroPython para Raspberry Pi Pico que
    encienda un LED cuando se mantenga presionado un botón.

    \bigbreak

    \textit{Solución:}

    \bigbreak

    El programa que se necesitará es sencillo:

    \begin{lstlisting}[language=Python, caption={Programa para el ejemplo 1}]
    import machine

    led = machine.Pin(18, machine.Pin.OUT)
    button = machine.Pin(0, machine.Pin.IN)

    while True:
        led.value(1) if button.value() == 1 else led.value(0)
    \end{lstlisting}

    Similar a como se programa en Arduino, primero se crearon dos variables
    para manejar la salida en el pin $18$ del LED y así también la entrada
    del pulsador en el pin $0$.

    \bigbreak

    Así, se puede construir el loop del programa el cual corre
    indefinidamente para ejecutar la lógica de programación. En este caso se
    define el valor ($1$ o $0$) del LED dependiendo del valor del pulsador en
    la entrada.

    \bigbreak

    El circuito se desarrollará de la siguiente manera:

    \begin{figure}[H]
        \centering
        \includegraphics[width=0.2\paperwidth]{images/wokwi-actions}
        \caption{Acciones disponibles en el menú de Wokwi}
    \end{figure}

    En la acción \say{Add a new part} se puede seleccionar un dispositivo
    para agregar al circuito.

    \begin{figure}[H]
        \centering
        \includegraphics[width=0.2\paperwidth]{images/wokwi-parts}
        \caption{Partes o dispositivos en Wokwi}
    \end{figure}

    De esa forma, se puede agregar cualquier dispositivo como LED,
    resistencia, etc., a fin de elaborar el circuito. El circuito de este
    ejemplo es como sigue:

    \begin{figure}[H]
        \centering
        \includegraphics[width=0.4\paperwidth]{images/wokwi-example-1-circuit}
        \caption{Circuito para el ejemplo 1}
    \end{figure}

    Notar que al hacer hover en un pin se muestra mayor información sobre ese
    pin. La conexión a tierra se hace simplemente con uno de los pines tierra
    de la tarjeta.

    \bigbreak

    Por último, hay una pestaña para configurar los atributos de los
    dispositivos que se han agregado al proyecto. Se puede modificar el valor
    de la resistencia así como cualquier atributo de cada parte o dispositivo.

    \begin{figure}[H]
        \centering
        \includegraphics[width=0.3\paperwidth]
        {images/wokwi-example-1-diagram-src-code}
        \caption{Código fuente del circuito}
    \end{figure}

    \subsection{Ejemplo 2}

    Desarrollar un programa que tome dos entradas $A$ y $B$ y tenga dos
    salidas $Z_1$ y $Z_2$ que computen la compuerta lógica $AND$ y $OR$ de
    las entradas respectivamente.

    \bigbreak

    \textit{Solución:}

    \bigbreak

    De la misma manera que se desarrolló el ejemplo anterior, se seguirán
    trabajando los restantes. Por tanto, tenemos el código y circuito para
    este ejemplo:

    \begin{lstlisting}[language=Python, caption={Programa para el ejemplo 2}]
    import machine

    A = machine.Pin(1, machine.Pin.IN)
    B = machine.Pin(2, machine.Pin.IN)
    Z0 = machine.Pin(12, machine.Pin.OUT)
    Z1 = machine.Pin(13, machine.Pin.OUT)

    while True:
        Z0.value(A.value() and B.value())
        Z1.value(A.value() or B.value())
    \end{lstlisting}

    Se crean las entradas y las salidas, y en el loop se calculan los valores
    lógicos para asignar a la salida.

    \begin{figure}[H]
        \centering
        \includegraphics[width=0.5\paperwidth]{images/wokwi-example-2-circuit}
        \caption{Circuito para el ejemplo 2}
    \end{figure}

    \begin{figure}[H]
        \centering
        \includegraphics[width=0.5\paperwidth]{images/wokwi-example-2-sim-1}
        \caption{Simulación 1 (OR)}
    \end{figure}

    \begin{figure}[H]
        \centering
        \includegraphics[width=0.5\paperwidth]{images/wokwi-example-2-sim-2}
        \caption{Simulación 2 (AND, OR)}
    \end{figure}

    Como se puede ver en las capturas, al tener uno o dos interruptores
    cerrados se obtiene un uno lógico en la salida del LED azul y al tener
    ambos interruptores cerrados ya se activa el LED verde.

    \subsection{Ejemplo 3}

    Diseñar un circuito de arranque y paro. Para ello se requiere de un
    pulsador normalmente cerrado para paro, uno normalmente abierto para
    arranque, un contactor y un contactor normalmente abierto. Este circuito
    consiste en un circuito lógico secuencial con la siguiente ecuación:

    $$
    Q = (A + Q)\bar{P}
    $$

    \bigbreak

    \textit{Solución:}

    \bigbreak

    En este caso, simplemente hay que calcular la fórmula en el programa y
    hacer un circuito similar al anterior pero con pulsadores. El código
    queda de la siguiente manera:

    \bigbreak

    \begin{lstlisting}[language=Python, caption={Programa para el ejemplo 3}]
    import machine

    A = machine.Pin(1, machine.Pin.IN)
    P = machine.Pin(2, machine.Pin.IN)
    Q = machine.Pin(13, machine.Pin.OUT)

    while True:
        value = (A.value() or Q.value()) and not(P.value())
        Q.value(value)
    \end{lstlisting}

    El circuito es el siguiente:

    \begin{figure}[H]
        \centering
        \includegraphics[width=0.5\paperwidth]{images/wokwi-example-3-circuit}
        \caption{Circuito para el ejemplo 3}
    \end{figure}

    Y al correr la simulación se puede arrancar con el botón verde y detener
    operaciones con el botón amarillo.

    \begin{figure}[H]
        \centering
        \includegraphics[width=0.5\paperwidth]{images/wokwi-example-3-sim-1}
        \caption{Simulación 1 (Arranque)}
    \end{figure}

    \begin{figure}[H]
        \centering
        \includegraphics[width=0.5\paperwidth]{images/wokwi-example-3-sim-2}
        \caption{Simulación 2 (Detener)}
    \end{figure}

    \section{Análisis de Resultados}\label{sec:análisis-de-resultados}


    Cada enunciado dado en cada ejemplo fue comprobado en el simulador al
    corre el experimento.

    \bigbreak

    Para el ejemplo 1 se encendió un LED de acuerdo a la entrada de un
    pulsador. Para el ejemplo 2 se obtuvieron las dos salidas $AND$ y $OR$
    para operar las dos entradas. Por último, para el ejemplo 3 se probó el
    sistema de arranque al enclavar la salida con el botón verde y reiniciar
    con el botón amarillo.

    \section{Conclusión}\label{sec:conclusion}

    Se desarrollaron tres ejemplos básicos sobre Raspberry Pi Pico,
    utilizando MicroPython para el código y Wokwi para la simulación.

    \printbibliography

\end{document}
